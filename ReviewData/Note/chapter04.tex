\chapter{控制项目范围}
\section{项目范围管理概述}
目前IT项目最大的问题是项目需求与范围的不确定性和易动性。
\subsection{项目范围与项目范围管理}
\subsubsection*{项目范围的定义}
项目范围是指产生项目产品阶段包括的所有工作及产生这些产品经过的所有过程。
\par 项目范围的定义:项目产品范围、项目工作范围。
\par 产品范围衡量标准:根据客户要求来进行;
\par 工作范围衡量标准:根据项目范围管理计划来检验。
\subsubsection*{项目范围管理的定义}
项目范围管理是指对项目包括什么与不包括什么的定义与控制过程。 
\par 主要任务:保证项目利益相关者在项目要产生什么样的可交付成果方面达成共识,也要在如何生产这些可交付成果方面达成共识。
\subsubsection*{项目范围管理的步骤} 
\begin{enumerate}
	\item 把客户的需求转变为对项目产品的定义。
	\item 根据项目目标与产品分解结构,把项目产品的定义转化为对项目工作范围的说明。
	\item 通过工作分解结构,定义项目工作范围。
	\item 项目干系人认可并接受项目范围。
	\item 授权与执行项目工作,并对项目进展进行控制。
\end{enumerate}
\subsection{项目范围管理的重要性}
\begin{enumerate}
	\item 提高费用、时间和资源估算的准确性。 
	\item 确定进度测量和控制的基准。 
	\item 有助于项目分工。
\end{enumerate}
\subsection{项目范围管理过程}
\begin{figure}[!h]
	\centering
	\includegraphics[width=0.8\textwidth]{image/4-1}
	\caption{IT项目范围管理过程}
\end{figure}
\begin{enumerate}
	\item 范围规划:制定项目范围管理计划,确定、核实与控制项目范围,定义WBS;
	\item 范围定义:制定详细的项目范围说明书,作为将来项目决策的根据;
	\item 制作WBS:将项目大的可交付成果与项目工作划分为较小和更易管理的组成部分;
	\item 范围核实:正式验收已经完成的项目可交付成果;
	\item 范围控制:通过对造成项目范围变更的因素施加影响,控制项目范围的变更。
\end{enumerate}
\section{项目启动}
项目启动最重要的成果:通过项目章程和确定项目经理。
\par 最好方法:目标驱动、结果引导。
\subsection{了解IT项目背景信息}
需要了解一些基本信息、技术信息。
\subsection{项目启动的依据}
依据:企业战略目标、项目选择的标准、项目建设的目的、项目成果说明书、与项目相关的历史资料。
\subsection{IT项目的启动会议}
目的:使项目干系人明确项目意义、背景、目标、范围、要求及各自应该承担的职责与拥有的权利。
\subsection{项目章程}
“项目章程” 是一个特别的文件形式,正式承认项目存在的重要文件,组织通过它来授权项目工作的正式开展,它可以是项目立项书、企业需求说明书、产品说明书、项目任务书、开工令或项目描述表。
\section{项目范围规划}
范围规划的任务:确定项目范围,明确项目的主要可交付成果,制定项目范围管理计划,记载如何确定、核实与控制项目范围,以及如何制定与定义WBS。
\subsection{范围规划的依据}
\begin{enumerate}
	\item 环境因素
	\item 组织过程资产
	\item 项目章程
	\item 项目初步范围说明书
\end{enumerate}
\subsection{项目范围管理计划}
项目范围管理计划是项目管理团队确定、记载、核实、管理和控制项目范围的指南。
\par 项目范围管理计划的主要内容:
\begin{itemize}
	\item 根据详细的项目范围说明书制作的WBS,
	\item 如何正式核实与验收项目已完成可交付成果
	\item 控制详细项目范围说明书变更请求处理的方式
\end{itemize}
\section{项目范围定义}
通过这一过程,将项目工作任务分解成易于操作和管理的工作单元。
\par 范围定义过程的输出是\textbf{项目工作分解结构(WBS,Work Breakdown Structure)}。
\subsection{范围定义概述}
范围定义就是定义项目的范围,即根据范围规划过程定义的范围管理计划,采取一定的方法,逐步得到精确的项目范围。
\par 3 个主要约束条件:质量、时间、成本。
\subsection{范围定义的依据}
主要依据是项目已有的文件和相关信息。
\subsection{IT项目范围说明书}
详细项目范围说明书是针对初步项目范围说明书而言的,该说明书详细地说明了项目产品或可交付成果及生成这些项目交付成果所要求的工作。
包括如下内容:
\begin{itemize}
	\item 项目目标和项目范围指标 
	\item 项目产品范围说明书 
	\item 项目可交付成果的规定 
	\item 项目约束条件和假定条件 
	\item 项目配置关系及其管理要求 
	\item 项目批准的规定 
\end{itemize}
\subsection{软件项目范围说明书}
软件需求规格说明书(Software Requirements Specifications,SRS),也称为功能规格说明、产品规格说明、需求文档或系统规格说明;
\par 精确地阐述一个软件系统必须提供的功能和性能以及它所要考虑的限制条件;不仅\par 是系统测试和用户文档的基础,也是所有子系列项目规划、设计和编码的基础。
\section{工作分解结构技术}
工作分解结构( Work Breakdown Structure,WBS)是一种为了便于管理和控制而将项目工作任务分解的技术。
\subsection{WBS的用途}
\begin{enumerate}
	\item 确定了项目整个范围,并将其有条理地、分层次地组织在一起;
	\item 属于工作分解结构底层组成部分的计划工作叫做 “工作细目”,可以安排在进度表中,用来估算费用,进行监视和控制;
	\item 工作分解结构是当前批准的项目范围说明书规定的工作。
\end{enumerate}
注:没有包含在WBS里的工作是不应该做的。
\subsection{制作WBS的方法}
制作WBS的方法:使用指导方针、类比法、由上至下法、由下至上法。
\subsection{WBS的应用}
基本原则:
\begin{enumerate}
	\item 一个单位工作任务在WBS中是唯一的;   
	\item 一个WBS项的工作内容是其下一级各项工作之和;    
	\item WBS中的每一项工作都明确由一个人负责;
	\item WBS必须与工作任务的实际执行过程相一致;
	\item 项目组成员必须参与WBS的制定;
	\item 每一个WBS项都必须归档;
	\item WBS要具有一定的灵活性以适应无法避免的变更需要。 
\end{enumerate}
\section{项目范围核实与控制}
\subsection{项目范围核实}
范围核实是指利益相关者对范围的正式接受。
\par IT项目范围核实的步骤:
\begin{enumerate}
	\item 确定需要进行范围核实的时间
	\item 识别范围核实需要哪些投入
	\item 确定范围正式被接受的标准和要素
	\item 确定范围核实会议的组织步骤
	\item 组织范围核实会议
\end{enumerate}
\subsection{项目范围的控制}
产生变更原因:需求不明确、系统实施时间过长、用户业务需求改变、系统正常升级。 
\par 项目范围控制是指当项目范围变化时对其采取纠正措施的过程,以及为使项目朝着目标方向发展而对项目范围进行调整的过程。
\par 应对变更措施:
\begin{itemize}
	\item 项目启动阶段的需求范围变更预防
	\item 项目实施阶段的需求范围变更
	\item 项目收尾阶段的总结
\end{itemize}
\subsection{软件项目范围变更控制}
变更控制的目的不是控制变更的发生,而是对变更进行管理,确保变更有序进行。
\section{软件项目范围控制的常见问题及对策}
\begin{enumerate}
	\item 与用户一起深入举行软件变更分析
	\item 妥善处理不合理的变更要求
	\item 正确处理用户说不清楚的需求
	\item 使用模板来管理软件项目变更
\end{enumerate}