\chapter{协调项目人力资源}
\section{人力资源本管理概述}
人力资源( Human Resource ,HR )
\subsection{项目人力资源}
人力资源通常指能够推动整个经济和社会发展的劳动者的能力,包括体力劳动和智力劳动。
\par 现代社会有5大可利用资源:物力资源、财力资源、信息资源、文化资源和人力资源。
\par 人力资源的特征:生物性、社会性、时效性、能动性、个体独立性、可再生性。
\par 人力资源是人类可用于生产产品或提供各种服务的活力、技能、知识和可提供的商誉价值。
\par\textbf{项目人力资源:}是指能推动整个项目发展的所有相关者的能力。
\subsection{项目人力资源管理}
\textbf{项目人力资源管理:}通过不断的获得人力资源,把得到的人力整合到项目中并融为一体,保持和激励他们对项目的忠诚和积极性,控制他们的工作绩效并作出相应的调整,尽量发挥他们的潜能,以支持项目目标的实现,这样的活动、职能、责任和过程叫项目人力资源管理。
\par 人力资源管理的目标:在于帮助组织吸引、保留和激励员工。
\par 同时,还具有如下几方面的意义:改善员工的工作生活质量、提高生产效率、获得竞争优势。
\par 人力资源管理的主要过程:
\begin{itemize}
	\item \textbf{人力资源规划:}根据项目管理计划和实际需求,对项目角色、职责以及报告关系进行识别、分配和归档。
	\item \textbf{项目团队组建:}根据项目人力资源规划,通过有效手段获得项目所需的人员,组建项目团队。
	\item \textbf{项目团队建设:}提高项目团队成员的技能,以加强他们完成项目任务的能力;增进团队成员之间信任感和凝聚力,以提高团队协作的能力,达到提高生产力的目的。
	\item \textbf{项目团队管理:}通过跟踪团队成员绩效,分析反馈信息,解决问题并协调各类变更,提高项目绩效。
\end{itemize}
\begin{figure}[!h]
	\centering
	\includegraphics[width=0.8\textwidth]{image/8-1}
	\caption{人力资源管理的主要过程}
\end{figure}
\subsection{项目人力资源管理的特点}
项目人力资源管理是组织人力资源管理的具体应用,项目人力资源管理必然要遵循组织人力资源管理的原理并实现相同性质的功能。
项目人力资源管理的特殊性,主要体现在:
\begin{itemize}
	\item 项目人力资源管理与组织人力资源管理的区别
	\item IT项目人力资源管理与一般人力资源管理的区别
\end{itemize}
\section{项目人力资源规划}
项目人力资源规划的目的:确定项目的角色、职责、报告关系,并制定人员配备管理计划。
\par 项目人力资源规划被作为项目\textbf{最初阶段的一项主要工作}来完成。
\subsection{IT项目组织的确定}
\begin{itemize}
	\item 针对项目的实际需求确认项目需要哪种类型的成员,是人力资源规划的关键活动之一;
	\item 应该根据IT项目的特点和实际项目的需求,以及已识别的项目角色、职责、报告关系,构建项目的组织结构图。 
\end{itemize}
\subsection{IT项目人员配备管理计划}
人员配备管理计划描述何时、以何种方式满足项目人力资源需求。
\par 人员配备计划因IT项目的规模和应用领域的差异而不同,但一些内容是必须包括的:项目团队组建、时间安排 、成员遣散安排、培训需求。
\section{项目团队组建}
\subsection{项目经理的选择}
项目团队组建的主要任务:根据项目资源规划的成果,获取完成项目工作所需的人力资源。
\par 项目管理的组织特征表明,IT项目成败的关键人物是项目经理,他在项目管理中起到决定性的作用。
\par 项目经理的选择的三种方式:
\begin{itemize}
	\item 由企业高层领导委派
	\item 由企业和用户协商选择
	\item 竞争上岗
\end{itemize}
\par IT项目经理至少需要具备三种基本能力:
\begin{itemize}
	\item 解读项目信息的能力
	\item 发现和整合项目资源的能力
	\item 将项目构想变成项目成果的能力
\end{itemize}
\subsection{项目团队成员的选择}
在面试IT项目团队候选人时应注意候选人是否具备以下几方面的能力:
\begin{itemize}
	\item 扎实的专业基础;
	\item 独立、创新的工作能力;
	\item 良好的沟通能力和团队合作精神;
	\item 认真、严谨的工作态度;
	\item 成就感强、工作有激情;
	\item 具备锲而不舍的精神;
	\item 善于总结和运用工作经验和教训。 
\end{itemize}
\section{项目团队的建设与管理}
\subsection{团队的概念}
团队是层次合理、分工明确、任务清晰、责任到位,能将有限资源最有效地整合的机构。
\par 项目团队能否有效地开展项目管理活动,体现在几个方面:
\begin{itemize}
	\item 拥有共同目标
	\item 合理分工与协作
	\item 具有高度的凝聚力
	\item 团队成员相互信任
	\item 能够有效的沟通
\end{itemize}
\subsection{项目团队的发展与建设}
团队专长=集体知识+规范的团队交流方式
\begin{enumerate}
	\item 认为团队成员都是最好的。
	\item 对事不对人,解决问题而不是责备人。
	\item 召开经常性的、有效的会议。
	\item 把每个工作组的人数限制在3到7人。
	\item 促进成员和其他的项目干系人更好地相互了解。
	\item 强调团队的同一性。
	\item 教育、培养项目组成员,鼓励他们互相帮助。
	\item 认可个人和团队的成绩。
\end{enumerate}
\subsection{项目人员培训}
IT项目培训是指为提高项目开发人员的技能和知识,增强项目开发能力,使员工能在现有项目和将来的岗位上胜任其角色而进行的一切有计划、有组织的学习和训练活动。
\par 建立良好的员工培训系统好处:
\begin{itemize}
	\item 确保获得组织和项目所需要的人才
	\item 留住人才
	\item 提高员工的成就感
\end{itemize}
人力资源部门一般提供3种类型的培训:技术培训、职向培训、文化培训。
\subsection{绩效评估}
绩效是指员工完成工作或履行职务的结果,即员工所创造的价值。
\\绩效具有以下特征:
\begin{enumerate}
	\item 绩效是一定的主体作用于一定的客体而表现出来的效用,即它是在工作过程中产生的。
	\item 绩效是人们行为的后果,是目标的完成程度,是客观存在的结果。
	\item 绩效必须具有实际效果,无效劳动的结果不能称为绩效。
	\item 绩效应当体现投入与产出的关系,即考虑效率的问题。
	\item 绩效应当有一定的可度量性。
\end{enumerate}
绩效评估的目的:
\begin{enumerate}
	\item 激励:通过正确评价员工的行为和绩效,给予员工恰当的激励。
	\item 培训:通过绩效评估可以发现员工所欠缺的技能和知识,从而设计具有针对性的培训,更好的提高员工的绩效。
	\item 沟通:绩效评估面谈,可以加强组织与员工之间的沟通和协调,为改进员工未来的绩效达成共识。
\end{enumerate}
绩效评估一般要经过4个步骤:
\begin{itemize}
	\item 制定绩效评估指标和标准
	\item 绩效评估过程
	\item 绩效评估面谈
	\item 绩效评估审核
\end{itemize}
7种绩效评估方法:
等级评定法、比例控制法、排序法、成对比较法、关键事件法、行为锚定等级法、目标管理法。
\section{项目人力资源的激励}
\subsection{动机理论}
动机,是指激励人去行动的主观原因,经常以愿望、兴趣、理想等形式表现出来,是个人发动和维持其行为,使其导向某一目标的一种心理状态。
\par 行为科学认为,人的动机来自需要,由需要确定人们的行为目标,激励则作用于人内心活动,激发、驱动和强化人的行为。
\par 常见的激励理论有:双因素理论 、期望理论、公平理论、强化理论。
