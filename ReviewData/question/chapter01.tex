\chapter{走进IT项目管理}
\section{项目的定义、特征}
项目是一个特殊的、将被完成的有限任务,它是在一定时间内,满足一系列特定目标的多项相关工作的总称。
\par 项目是为完成某一独特的产品和服务所做的一次性努力。(另一说法)
\par 项目的特征:目标的独特性、项目的一次性、项目的整体性、项目的临时性、项目的不确定性、资源的多变性等。
\section{判断下面哪些活动是项目}
每周一去上项目管理课、\underline{一次野餐活动}、\underline{集体婚礼}、社区安保、\underline{开发图书馆信息管理系统}、 每天的卫生保洁、\underline{制定神州飞船计划}。
\section{PMI:美国项目管理协会}
项目管理知识体系把项目管理划分为9个知识领域和44个管理过程。
\par 9个知识领域包括4个核心知识领域、4个辅助知识领域和1个项目整体管理。
\par 4个核心知识领域,之所以称其为核心知识领域是因为这几个方面将形成具体的项目目标,核心知识领域分别是:
\par 
\section{什么是项目?什么是项目管理?导致IT项目失败的主要原因是什么?}
\section{什么是软件项目?如何分类?软件项目的特点主要表现在哪些方面?}